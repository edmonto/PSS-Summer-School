\documentclass[11pt]{article}


%%%%%%%%%%%%%%%%%%%%%%%%%%%%%%%%%%%%%%%% DO NOT EDIT BELOW THIS %%%%%%%%%%%%%%%%%%%%%%%%%%%%%%%%%%%%%%%%
\usepackage[margin=1in]{geometry}
\linespread{1.5}

\begin{document}

%%% Header %%%
\begin{center}
{\large 2017 PSS SUMMER SCHOOL}

Assignment 1


\textbf{Eduard Danalache}

\textbf{Rice University}

\textit{Due: 12:30pm on June 22}



\end{center}

\vspace{0.5cm}
%%%%%%%%%%%%%%%%%%%%%%%%%%%%%%%%%%%%%%%% DO NOT EDIT ABOVE THIS %%%%%%%%%%%%%%%%%%%%%%%%%%%%%%%%%%%%%%%%


%%% Body %%%
\section{Problem 2.1}
\subsection{a.}
\begin{itemize}
\item \textbf{str(obj, ...):} Gives the structure of an R object $obj$. Particularly useful for displaying the contents of lists. Does not return anything, but instead prints output to the terminal.
\item \texttt{is.vector(x, mode="any"):} Used to determine whether an object $x$ is a vector with type $mode$. If $mode$ == "any" then returns TRUE if $x$ is any type of vector. Otherwise returns TRUE if typeof($x$) == $mode$, FALSE otherwise.
\item \textit{is.atomic(x):} Returns TRUE if object $x$ is of an atomic type, FALSE otherwise.
\item \textsl{is.list(x):} Returns TRUE if object $x$ is a list or pairlist of length $>$ 0.
\end{itemize}
\subsection{b.}
\begin{itemize}
\item \textsc{typeof(x):} Determines the type or storage mode of object $x$. Returns a character string with the type or mode.
\item \textsf{is.numeric(x):} Returns TRUE if object $x$ is of mode numeric (i.e. it has type integer or double), FALSE otherwise.
\item \textsc{is.integer(x):} Returns TRUE if object $x$ is of type integer, FALSE otherwise.
\item \textsc{is.double(x):} Returns TRUE if object $x$ is of type double, FALSE otherwise.
\item \textsc{is.character(x):} Returns TRUE if object $x$ is of type character, FALSE otherwise.
\item \textsc{is.logical(x):} Returns TRUE if object $x$ is of type logical, FALSE otherwise.
\end{itemize}

\end{document}